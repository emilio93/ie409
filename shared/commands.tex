% Copyright 2017 Emilio Rojas
%
% Permission is hereby granted, free of charge, to any person obtaining a copy of
% this software and associated documentation files (the "Software"), to deal in
% the Software without restriction, including without limitation the rights to
% use, copy, modify, merge, publish, distribute, sublicense, and/or sell copies of
% the Software, and to permit persons to whom the Software is furnished to do so,
% subject to the following conditions:
%
% The above copyright notice and this permission notice shall be included in all
% copies or substantial portions of the Software.
%
% THE SOFTWARE IS PROVIDED "AS IS", WITHOUT WARRANTY OF ANY KIND, EXPRESS OR
% IMPLIED, INCLUDING BUT NOT LIMITED TO THE WARRANTIES OF MERCHANTABILITY, FITNESS
% FOR A PARTICULAR PURPOSE AND NONINFRINGEMENT. IN NO EVENT SHALL THE AUTHORS OR
% COPYRIGHT HOLDERS BE LIABLE FOR ANY CLAIM, DAMAGES OR OTHER LIABILITY, WHETHER
% IN AN ACTION OF CONTRACT, TORT OR OTHERWISE, ARISING FROM, OUT OF OR IN
% CONNECTION WITH THE SOFTWARE OR THE USE OR OTHER DEALINGS IN THE SOFTWARE.

% Unnumbered section added to toc
% 1: The section's name.
\newcommand{\usection}[1]
{
  \section*{#1}
  \addcontentsline{toc}{section}{#1}
}

% Unnumbered subsection added to toc
% 1: The subsection's name.
\newcommand{\usubsection}[1]
{
  \subsection*{#1}
  \addcontentsline{toc}{subsection}{#1}
}

% Unnumbered subsubsection added to toc
% 1: The subsubsection's name.
\newcommand{\usubsubsection}[1]
{
  \subsubsection*{#1}
  \addcontentsline{toc}{subsubsection}{#1}
}

% Circuitikz equal
\newcommand*{\equal}{=}

% Adds a title page to the document
% 1: The document's name
% 2: The document's owner's name
% 3: The document's owner's id
% 4: The document's date

% \portada{Homework}{John Doe}{123456}{\today}
\newcommand{\portada}[4]
{
  \begin{titlepage}
    \vspace*{\fill}
    \begin{center}

      \textsc{\Large Universidad de Costa Rica}\\
      [0.25in]

      \textsc{\Large Escuela de Ingeniería Eléctrica}\\
      [0.55in]

      \textsc{\Large Análisis de Sistemas - IE0409}\\
      [0.25in]

      \textsc{\huge \bfseries {#1}}\\
      [0.5in]

      \textsc{\large {#2}}\\
      \textsc{\large {#3}}\\
      [0.5in]

      \textsc{\large {#4}}
    \end{center}
    \vspace*{\fill}
  \end{titlepage}
}

% \portadapequena{Homework}{John Doe}{123456}{\today}
\newcommand{\portadapequena}[4]
{
  \begin{flushleft}
    {\large Universidad de Costa Rica}\\
    {\large Escuela de Ingeniería Eléctrica}\\
    {\large Análisis de Sistemas - IE0409}\\
    {\large \bfseries {#1}}\\
    {\large {#2}} -
    {\large {#3}}\\
    {\large {#4}}
  \end{flushleft}
}

% Command "alignedbox{}{}" for a box within an align environment
% Source: http://www.latex-community.org/forum/viewtopic.php?f=46&t=8144
\newlength\dlf  % Define a new measure, dlf
\newcommand\alignedbox[2]{
% Argument #1 = before & if there were no box (lhs)
% Argument #2 = after & if there were no box (rhs)
&  % Alignment sign of the line
{
\settowidth\dlf{$\displaystyle #1$}
    % The width of \dlf is the width of the lhs, with a displaystyle font
\addtolength\dlf{\fboxsep+\fboxrule}
    % Add to it the distance to the box, and the width of the line of the box
\hspace{-\dlf}
    % Move everything dlf units to the left, so that & #1 #2 is aligned under #1 & #2
\boxed{#1 #2}
    % Put a box around lhs and rhs
}
}
