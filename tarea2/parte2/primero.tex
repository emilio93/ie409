\usubsection{1)}
\begin{align*}
  3\frac{d^3y(t)}{dt^3}
  + 5 \frac{d^2y(t)}{dt^2}
  + 26 \frac{dy(t)}{dt}
  + 3y(t)
  &= 6\frac{du(t)}{dt}
  + 4u(t)
  \\
  \left(
    3p^3 + 5 p^2 + 26 p + 3
  \right) y(t)
  &= \left(
    6p + 4
  \right) u(t)
  \\
  y(t) &= \left( \frac{6p + 4}{3p^3 + 5 p^2 + 26 p + 3} \right) u(t)
\end{align*}

Las raices para el denominador son $\approx -0.1179,  -0.7744\pm j2.8079$.
\begin{align*}
  y(t) &= \left( \frac{6p + 4}{(p+0.1179)(p+0.7744-j2,8079)(p+0.7744+j2,8079)} \right) u(t)
  \\
  y(t) &= \left( \frac{6p + 4}{(p+0.1179)(p^2+1.5488p+8.4840)^2} \right) u(t)
\end{align*}

Se encuentran las fracciones parciales:

\begin{align*}
  \frac{6p + 4}{(p+0.1179)(p^2+1.5488p+8.4840)^2}
  &= \frac{A}{p+0.1179} + \frac{B}{p+0.7744-j2,8079} + \frac{C}{p+0.7744+j2,8079}
\end{align*}
\begin{align*}
  6p + 4
  &= A (p^2+1.5488p+8.4840)^2
  + B (p+0.1179)(p+0.7744+j2,8079)
  + C (p+0.1179)(p+0.7744-j2,8079)
\end{align*}

Cuando $p=-0.1179$:
\begin{align*}
  6(-0.1179) + 4
  &= A ((-0.1179)^2+1.5488(-0.1179)+8.4840)^2
  \\
  A &= \frac{6(-0.1179) + 4}{((-0.1179)^2+1.5488(-0.1179)+8.4840)^2}
  \\
  A &\approx 0.0476
\end{align*}

Cuando $p=-0.7744+j2,8079$:
\begin{align*}
  6(-0.7744+j2,8079) + 4
  &= B ((-0.7744+j2,8079)+0.1179)((-0.7744+j2,8079)+0.7744+j2,8079)
  \\
  B &= \frac{6(-0.7744+j2,8079) + 4}
  {((-0.7744+j2,8079)+0.1179)((-0.7744+j2,8079)+0.7744+j2,8079)}
  \\
  B &= \frac{-0.6464+j16.8474}
  {(-0.6565+j2,8079)(j5.6158)}
  = \frac{-0.6464+j16.8474}
  {-15.7686-j3.6868}
  \\
  B &\approx -0.1980-j1.0221
\end{align*}

Cuando $p=-0.7744-j2.8079$:
\begin{align*}
  6(-0.7744-j2.8079) + 4
  &= C ((-0.7744-j2.8079)+0.1179)((-0.7744-j2.8079)+0.7744-j2,8079)
  \\
  C &= \frac{6(-0.7744-j2.8079) + 4}
  {((-0.7744-j2.8079)+0.1179)((-0.7744-j2.8079)+0.7744-j2,8079)}
  \\
  C &= \frac{-0.6464-j16.8474}
  {(-0.6565-j2,8079)(-j5.6158)}
  = \frac{-0.6464-j16.8474}
  {-15.7686+j3.6868}
  \\
  C &\approx -0.1980+j1.0221
\end{align*}

Entonces se tiene:
\begin{align*}
  y(t) &= \left(
    \frac{0.0476}{p+0.1179}
    + \frac{-0.1980-j1.0221}{p+0.7744-j2,8079}
    + \frac{-0.1980+j1.0221}{p+0.7744+j2,8079}
  \right) u(t)
\end{align*}

Se asignan las variables de estado:
\begin{align*}
  x_1 &= \frac{1}{p+0.1179}u \Rightarrow \dot{x}_1 = -0.1179x_1 + u
  \\
  x_2 &= \frac{1}{p+0.7744-j2,8079}u \Rightarrow \dot{x}_2 = -(0.7744-j2,8079)x_2 + u
  \\
  x_3 &= \frac{1}{p+0.7744+j2,8079}u \Rightarrow \dot{x}_3 = -(0.7744+j2,8079)x_3 + u
\end{align*}

\usubsubsection{Modelo en Variables de Estado}
\[ \setstretch{2}
  \dot{\textbf{x}}=
  \left[
    {\begin{array}{ccc}
      -0.1179 & 0                 & 0 \\
      0       & -(0.7744-j2,8079) & 0 \\
      0       & 0                 & -(0.7744+j2,8079) \\
    \end{array}}
  \right]
  \textbf{x} +
  \left[
    {\begin{array}{cc}
      1 \\
      1 \\
      1 \\
    \end{array}}
  \right]
  \textbf{u}
\]

\[ \setstretch{2}
  \textbf{y}=
  \left[
    {\begin{array}{ccc}
      0.0476 & -0.1980-j1.0221     & -0.1980+j1.0221 \\
    \end{array}}
  \right]
  \textbf{x} +
  0
  \textbf{u}
\]
