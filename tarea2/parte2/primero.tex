\usubsection{1)}
\begin{align*}
  3\frac{d^3y(t)}{dt^3}
  + 5 \frac{d^2y(t)}{dt^2}
  + 26 \frac{dy(t)}{dt}
  + 3y(t)
  &= 6\frac{du(t)}{dt}
  + 4u(t)
  \\
  \left(
    p^3 + \frac{5}{3} p^2 + \frac{26}{3} p + 1
  \right) y(t)
  &= \left(
    2p + \frac{4}{3}
  \right) u(t)
  \\
  y(t) &= \left( \frac{2p + \frac{4}{3}}{p^3 + \frac{5}{3} p^2 + \frac{26}{3} p + 1} \right) u(t)
\end{align*}

Las raices para el denominador son:
\begin{align*}
  p_1 &\approx -0.117867 &
  p_2 &\approx -0.774400 + j2.807921 &
  p_3 &\approx -0.774400 - j2.807921
\end{align*}
\begin{align*}
  y(t) &= \left( \frac{2p + \frac{4}{3}}{(p-p_1)(p-p_2)(p-p_3)} \right) u(t)
\end{align*}

Se encuentran las fracciones parciales:
\begin{align*}
  \frac{2p + \frac{4}{3}}{(p-p_1)(p-p_2)(p-p_3)}
  &= \frac{A}{p-p_1} + \frac{B}{p-p_2} + \frac{C}{p-p_3}
\end{align*}
\begin{align*}
  2p + \frac{4}{3}
  &= A (p-p_2)(p-p_3)
  + B (p-p_1)(p-p_3)
  + C (p-p_1)(p-p_2)
\end{align*}

Cuando $p=p_1$:
\begin{align*}
  2p_1 + \frac{4}{3}
  &= A (p_1-p_2)(p_1-p_3)
  \\
  A &= \frac{2p_1 + \frac{4}{3}}{(p_1-p_2)(p_1-p_3)}
  \\
  A &\approx -0.037799
\end{align*}

Cuando $p=p_2$:
\begin{align*}
  2p_2 + \frac{4}{3}
  &= B (p_2-p_1)(p_2-p_3)
  \\
  B &= \frac{2p_2 + \frac{4}{3}}{(p_2-p_1)(p_2-p_3)}
  \\
  B &\approx 0.018899-j0.479266
\end{align*}

Cuando $p=p_2$:
\begin{align*}
  2p_3 + \frac{4}{3}
  &= C (p_3-p_1)(p_3-p_2)
  \\
  C &= \frac{2p_3 + \frac{4}{3}}{(p_3-p_1)(p_3-p_2)}
  \\
  C &\approx 0.018899+j0.479266
\end{align*}

Entonces se tiene:
\begin{align*}
  y(t) &\approx \left(
    \frac{-0.037799}{p + 0.117867}
    + \frac{0.018899 - j0.479266}{p + 0.774400 - j2.807921}
    + \frac{0.018899 + j0.479266}{p + 0.774400 + j2.807921}
  \right) u(t)
\end{align*}

Se asignan las variables de estado:
\begin{align*}
  x_1 &\approx \frac{1}{p + 0.117867}u
  \Rightarrow \dot{x}_1 \approx -0.117867x_1 + u
  \\
  x_2 &\approx \frac{1}{p + 0.774400 - j2.807921}u
  \Rightarrow \dot{x}_2 \approx -(0.774400 - j2.807921)x_2 + u
  \\
  x_3 &\approx \frac{1}{p + 0.774400 + j2.807921}u
  \Rightarrow \dot{x}_3 \approx -(0.774400 + j2.807921)x_3 + u
\end{align*}

\usubsubsection{Modelo en Variables de Estado}
\[ \setstretch{2}
  \dot{\textbf{x}}\approx
  \left[
    {\begin{array}{ccc}
      -0.117867 & 0                       & 0 \\
      0         & -(0.774400 - j2.807921) & 0 \\
      0         & 0                       & -(0.774400 + j2.807921) \\
    \end{array}}
  \right]
  \textbf{x} +
  \left[
    {\begin{array}{cc}
      1 \\
      1 \\
      1 \\
    \end{array}}
  \right]
  \textbf{u}
\]

\[ \setstretch{2}
  \textbf{y}\approx
  \left[
    {\begin{array}{ccc}
      -0.037799 & 0.018899-j0.479266 & 0.018899+j0.479266 \\
    \end{array}}
  \right]
  \textbf{x} +
  \left[
    {\begin{array}{c}
      0
    \end{array}}
  \right]
  \textbf{u}
\]
